\documentclass[conference]{IEEEtran}
\IEEEoverridecommandlockouts
% The preceding line is only needed to identify funding in the first footnote. If that is unneeded, please comment it out.
\usepackage{cite}
\usepackage{amsmath,amssymb,amsfonts}
\usepackage{algorithmic}
\usepackage{graphicx}
\usepackage{textcomp}
\usepackage{xcolor}
\def\BibTeX{{\rm B\kern-.05em{\sc i\kern-.025em b}\kern-.08em
    T\kern-.1667em\lower.7ex\hbox{E}\kern-.125emX}}
\begin{document}

\title{}


\author{\IEEEauthorblockN{1\textsuperscript{st} Lía García-Pérez}
\IEEEauthorblockA{\textit{Dept. Arquitectura de computadores y automática} \\
\textit{Facultad de Ciencias Físicas. Universidad Complutense}\\
Madrid, Spain \\
liagar05@ucm.es}
\and
\IEEEauthorblockN{2\textsuperscript{nd} Juan F. Jiménez}
\IEEEauthorblockA{\textit{Dept. Arquitectura de computadores y automática} \\
\textit{Facultad de Ciencias Físicas. Universidad Complutense}\\
	Madrid, Spain \\
email address or ORCID}
}

\maketitle

\begin{abstract}

\end{abstract}

\begin{IEEEkeywords}

\end{IEEEkeywords}

\section{Introduction}

\section{Boat model}

Considering $R$ the rotation matrix, $E$ the  $90$ degrees rotation matrix, $F$ the applied force, $u_{x}$ is an unitary vector in the x axis direction refered to the body coordinates system, $\omega_{t}$ is the water velocity, $\mu_{a}$ is the resistance to rotation, $\omega$ is the angular velocity, $u(,p,\dot{p})$ the control signal commanded to the boat.

\begin{equation}
\mu=\begin{pmatrix}
\mu_{x} &0 \\ 0 &\mu_{y}
\end{pmatrix}
\end{equation}

\begin{equation}
R=\begin{pmatrix}
\cos\theta & \sin\theta \\ -\sin\theta & \cos\theta 
\end{pmatrix}
\end{equation}

\begin{equation}
E=\begin{pmatrix}
0 & 1 \\
-1 & 0
\end{pmatrix}
\end{equation}

\begin{equation}
\dot{p}=F\cdot R \cdot \mu^{-1} \cdot u_{x} + \omega_{t}
\end{equation}\label{eq1}

\begin{equation}
\omega = \frac{u(\dot{p},p)}{\mu_{a}}
\end{equation}\label{eq2}

\begin{equation}
\dot{R}=R \cdot E \cdot \omega
\end{equation}\label{eq3}


\subsection{Control function}
Lyapunnov function
\begin{equation}
V_{2}( \nu)=1-\hat{\dot{p_{d}^{t}}}\cdot \hat{\dot{p}}
\end{equation}

\begin{equation}
\frac{dV_{2}}{dt}=-\frac{d}{dt}(\hat{\dot{p_{d}^{t}}})\cdot \hat{\dot{p}}-\hat{\dot{p_{d}^{t}}}\cdot \frac{d}{dt}(\hat{\dot{p}})
\end{equation}\label{eq4}

We know that \cite{b1}:

\begin{equation}
\hat{\ddot{p_{d}}}=-\dot{\chi}_{d}\cdot E \cdot \hat{\dot{p_{d}}}
\end{equation}\label{eq5}

To compute $\hat{\ddot{p}}$ we begin with equation $18$ in \cite{b1}:

\begin{equation}
\hat{\ddot{p}}=\frac{-1}{\lVert\dot{p}\rVert}\cdot E\cdot \hat{\dot{p}}\cdot \hat{\dot{p}}^{t}\cdot E \cdot \ddot{p}
\end{equation}\label{eq6}

We now use the boat model to compute the time derivative from equation \ref{eq1}, and using \ref{eq2} and \ref{eq3} we obtain:

\begin{equation}
\ddot{p}=F\cdot R \cdot E \cdot \frac{u(\dot{p},p)}{\mu_{a}}\cdot \mu^{-1} \cdot u_{x}
\end{equation} \label{eq7}

Replacing in equation \ref{eq6} the value of $\ddot{p}$ computed in \ref{eq7} we obtain:

\begin{equation}
\hat{\ddot{p}}=\frac{-1}{\lVert\dot{p}\rVert}\cdot E\cdot \hat{\dot{p}}\cdot \hat{\dot{p}}^{t}\cdot E \cdot F\cdot R \cdot E \cdot \frac{u(\dot{p},p)}{\mu_{a}}\cdot \mu^{-1} \cdot u_{x}
\end{equation}\label{eq8}

In 2D $E\cdot R = R \cdot E$. Using this fact and $E\cdot E = -\mathbb{1}$ we can simplify equation \ref{eq8} to:

\begin{equation}
\hat{\ddot{p}}=\frac{F\cdot u(\dot{p},p)}{\lVert\dot{p}\rVert \cdot \mu_{a}}\cdot E\cdot \hat{\dot{p}}\cdot \hat{\dot{p}}^{t} \cdot R \cdot \mu^{-1} \cdot u_{x}
\end{equation} \label{eq9}

We can define

\begin{equation}
\phi = \hat{\dot{p}}^{t} \cdot R \cdot \mu^{-1} \cdot u_{x}
\end{equation} \label{eq10}

Note that equation \ref{eq10} is not an angle (as it was if the vehicle mode used is an unicycle) because $R \cdot \mu^{-1} \cdot u_{x}$ is not unitary.  $R \cdot \mu^{-1} \cdot u_{x}$ Is the vector that results for applying the water resistance to the x-axis unitary vector once rotated to the global coordinated system.

Using the defined $\phi$ we can replace all the terms in equation \ref{eq4}:

\begin{equation}
\frac{dV_{2}}{dt}=-(-\dot{\chi}_{d}\cdot E \cdot \hat{\dot{p_{d}}})^{t}\cdot \hat{\dot{p}}-\hat{\dot{p_{d}}}^{t}\cdot \frac{F\cdot u(\dot{p},p)}{\lVert\dot{p}\rVert \cdot \mu_{a}}\cdot E\cdot \hat{\dot{p}}\cdot \phi
\end{equation} \label{eq11}

Working out equation \ref{eq11}

\begin{equation}
\frac{dV_{2}}{dt}=-\dot{\chi}_{d}\cdot \hat{\dot{p_{d}}}^{t} \cdot E \cdot \hat{\dot{p}}- \frac{F\cdot u(\dot{p},p)\cdot \phi}{\lVert\dot{p}\rVert \cdot \mu_{a}}\cdot \hat{\dot{p_{d}}}^{t}\cdot E\cdot \hat{\dot{p}}
\end{equation} \label{eq12}

Taking a common factor:

\begin{equation}
\frac{dV_{2}}{dt}=(-\dot{\chi}_{d} -\frac{F\cdot u(\dot{p},p)\cdot \phi}{\lVert\dot{p}\rVert \cdot \mu_{a}})\cdot \hat{\dot{p_{d}}}^{t}\cdot E\cdot \hat{\dot{p}}
\end{equation} \label{eq13}

We can define our control function as:

\begin{equation}
 u(\dot{p},p)= \frac{\lVert\dot{p}\rVert \cdot \mu_{a}}{F\cdot \phi}\cdot (\dot{\chi}_{d}+K_{d}\cdot(\hat{\dot{p_{d}}}^{t}\cdot E\cdot \hat{\dot{p}}))
\end{equation} \label{eq14}

If we use the definition of equation \ref{eq14} and replace it in equation \ref{eq13} we obtain that:

\begin{equation}
\frac{dV_{2}}{dt}=- K_{d}\cdot(\hat{\dot{p_{d}}}^{t}\cdot E\cdot \hat{\dot{p}})^{2}
\end{equation} 


Which is decreasing.





\begin{thebibliography}{00}
\bibitem{b1} De Marina, Hector Garcia, et al. ``Guidance algorithm for smooth trajectory tracking of a fixed wing UAV flying in wind flows.'' 2017 IEEE international conference on robotics and automation (ICRA). IEEE, 2017.

\end{thebibliography}


\end{document}
